\documentclass[../main.tex]{subfiles}
\begin{document}

\chapter{Template}
\label{Chapter8} % Change X to a consecutive number; for referencing this chapter elsewhere, use \ref{ChapterX}
\section{Main Section 1}

\begin{figure}
\centering
    \begin{subfigure}[t]{.3\columnwidth}
      \includegraphics[width=\columnwidth]{Figures/Electron.pdf}
      \caption{I\label{fig:ta}}
    \end{subfigure}
    \begin{subfigure}[t]{.3\columnwidth}
      \includegraphics[width=\columnwidth]{Figures/Electron.pdf}
      \caption{II\label{fig:tb}}
    \end{subfigure}
    \caption{Main Caption}
  \begin{tikzpicture}[overlay,thick]
      %GRID
       %\draw [red] (-4,1) grid (4,17.5);
       %\draw[help lines,xstep=.5,ystep=.5] (-4,1) grid (4,17.5);
       %\foreach \x in {-4,-3,...,4} { \node [anchor=north] at (\x,1) {\x}; }
       %\foreach \y in {1,2,...,17.5} { \node [anchor=east] at (-4,\y) {\y}; }
       %POLYMERS
    \end{tikzpicture}
    \label{fig:tc}
\end{figure}

the. Why

%-----------------------------------
%	SUBSECTION 1
%-----------------------------------
\subsection{Subsection 1}

Test text subsection

%-----------------------------------
%	SUBSECTION 2
%-----------------------------------

\subsection{Subsection 2}
Test2

%----------------------------------------------------------------------------------------
%	SECTION 2
%----------------------------------------------------------------------------------------

\section{Main Section 2}
Test3

\end{document}
