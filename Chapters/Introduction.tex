% Chapter 1
\documentclass[../main.tex]{subfiles}


\begin{document}
\chapter{Introduction} % Main chapter title

\label{Chapter1} % For referencing the chapter elsewhere, use \ref{Chapter1} 

%----------------------------------------------------------------------------------------

% Define some commands to keep the formatting separated from the content 
\newcommand{\keyword}[1]{\textbf{#1}}
\newcommand{\tabhead}[1]{\textbf{#1}}
\newcommand{\code}[1]{\texttt{#1}}
\newcommand{\file}[1]{\texttt{\bfseries#1}}
\newcommand{\option}[1]{\texttt{\itshape#1}}

%----------------------------------------------------------------------------------------




\section{Section 1}

\lipsum[1-3]



\section{Section 2}

\lipsum[1-5]


\section{Section 3}\label{section:label_section_3}

\lipsum[2-4]
Look up here~\cite{Zorin:22} and check Fig.\ref{fig:4-a} and Fig.\ref{fig:4-b} (i.e. Fig.\ref{fig:fig_1-1}).

\begin{figure}[ht]
\centering
    \begin{subfigure}[t]{.59\columnwidth}
      \includegraphics[height=4.7cm]{example-image-a}
      \caption{Sub-caption 1}
    \end{subfigure}
    \begin{subfigure}[t]{.39\columnwidth}
      \includegraphics[height=4.7cm]{example-image-b}
      \caption{Sub-caption 2}
    \end{subfigure}
    \caption{Caption 1.}
  \begin{tikzpicture}[overlay,thick]
      %GRID
      %\draw [red] (-4,1) grid (4,17.5);
      %\draw[help lines,xstep=.5,ystep=.5] (-4,1) grid (4,17.5);
      %\foreach \x in {-4,-3,...,4} { \node [anchor=north] at (\x,1) {\x}; }
      %\foreach \y in {1,2,...,17.5} { \node [anchor=east] at (-4,\y) {\y}; }
      %POLYMERS
    \end{tikzpicture}
    \label{fig:fig_1-1}
\end{figure}

\lipsum[1-2]


\subsection{Sub-section 1}

\lipsum[1-2]

\section{Section 4}

\lipsum[1-3]


\subsection{Sub-section 1}

\lipsum[2-3]

\end{document}