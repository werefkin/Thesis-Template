\documentclass[../main.tex]{subfiles}
\newcommand{\vect}[1]{\mathrm{\boldsymbol{#1}}}

\begin{document}

\chapter{Lasers are super}
\label{ch:Chapter2} % Change X to a consecutive number; for referencing this chapter elsewhere, use \ref{ChapterX}
%flexible universal tools

\section{Introduction}

\lipsum[2-4]

\begin{figure}[ht]
\centering
    \begin{subfigure}[t]{.59\columnwidth}
      \includegraphics[height=4.7cm]{example-image-a}
      \caption{Data adapted from~\cite{Zorin:22} \label{fig:a}}
    \end{subfigure}
    \begin{subfigure}[t]{.39\columnwidth}
      \includegraphics[height=4.7cm]{example-image-b}
      \caption{Data adapted from~\cite{Zorin:22}\label{fig:b}}
    \end{subfigure}
    \caption{Caption 2.}
  \begin{tikzpicture}[overlay,thick]
      %GRID
      %\draw [red] (-4,1) grid (4,17.5);
      %\draw[help lines,xstep=.5,ystep=.5] (-4,1) grid (4,17.5);
      %\foreach \x in {-4,-3,...,4} { \node [anchor=north] at (\x,1) {\x}; }
      %\foreach \y in {1,2,...,17.5} { \node [anchor=east] at (-4,\y) {\y}; }
      %POLYMERS
    \end{tikzpicture}
    \label{fig:fig_2-1}
\end{figure}


\section{Section 2}

\lipsum[1-3]

Look up here~\cite{Zorin:22}.

% Please add the following required packages to your document preamble:

% Please add the following required packages to your document preamble:
% \usepackage{booktabs}
% Please add the following required packages to your document preamble:
% \usepackage{graphicx}
{\renewcommand{\arraystretch}{1.2}
\begin{table}[ht]
\centering
\caption{Dummy table.}
\label{tab:nolinearnmech}
\resizebox{\textwidth}{!}{%
\begin{tabular}{ccc}\hline\hline
 & \multicolumn{2}{c}{Dispersion at pump wavelength\vspace{1pt}} \\ \cline{2-3} 
\multicolumn{1}{c|}{Pump pulse duration} & \multicolumn{1}{c|}{Normal} & \multicolumn{1}{c|}{Anomalous} \\ \hline
\multicolumn{1}{|c|}{\begin{tabular}[c]{@{}c@{}}Picosecond-nanosecond \\ regime\end{tabular}} & \multicolumn{1}{c|}{\begin{tabular}[c]{@{}c@{}}Raman-scattering\\ Four-wave mixing \\Self-phase modulation\end{tabular}} & \multicolumn{1}{c|}{\begin{tabular}[c]{@{}c@{}}Modulation instability\\ Raman scattering\\ Four-wave mixing\\ Soliton fission\\ Soliton self-freqency shift\\ Dispersive wave generation\end{tabular}} \\ \hline
\multicolumn{1}{|c|}{\begin{tabular}[c]{@{}c@{}}Femtosecond-picosecond\\ regime\end{tabular}} & \multicolumn{1}{c|}{\begin{tabular}[c]{@{}c@{}}Raman-scattering\\ Four-wave mixing \\ \\ Self-phase modulation\\ Optical wave breaking\\ Four-wave mixing\end{tabular}} & \multicolumn{1}{c|}{\begin{tabular}[c]{@{}c@{}}Self-phase modulation\\ Soliton fission\\ Soliton self-freqency shift\\ Raman scattering\\ Four-wave mixing\\ Modulation instability\\ Dispersive wave generation\end{tabular}} \\ \hline
\multicolumn{1}{l|}{}   & \multicolumn{1}{c|}{\begin{tabular}[c]{@{}c@{}}Self-phase modulation\\ Short pump pulses (fs)\\ Self-seeded generation (coherent)\end{tabular}} & \multicolumn{1}{c|}{\begin{tabular}[c]{@{}c@{}}Modulation instability\\ Long pump pulses” (ns)\\ Noise-seeded (incoherent)\end{tabular} }\\ \hline \hline
\end{tabular}%
}
\end{table}
}

The most commonly used dummy form of the time-domain GNLSE employed to describe~\textendash~with respect to supercontinuum generation~\textendash~the evolution of the complex electrical field envelope\footnote{Dummy footnote.} $A(z,t)$~\cite{Zorin:22} is defined as:

\begin{equation}
\begin{gathered}
\cfrac{\partial A}{\partial z}=-\cfrac{\alpha}{2}A - \sum\limits_{k\geq2}\cfrac{i^{k+1}}{k!} \beta_k \cfrac{\partial^k A}{\partial T^k}+i\gamma \left(1+\cfrac{1}{\omega_0}\cfrac{\partial}{\partial T} \right)\\
\times \left(A(z,T) \int_{-\infty}^{\infty} R(T') \left|A(z,T-T')\right|^2 dT'\right),
\end{gathered}
\label{eq:gnlse}
\end{equation}
where $z$ is the travel distance (along fiber's length), $\omega_0$ is the carrier frequency (i.e. center frequency of the input pulse), the change of variable $T=t-\beta_1 z$ is used to define co-moving retarded frame with group velocity of the reference wavelength, $t$ is time.
The first and second terms on the right-hand side Eq.~\eqref{eq:gnlse} represents linear propagation effects, i.e. loss and dispersion respectively, where $\alpha$ is fiber losses and $\beta_k$ are the dispersion coefficients. The third term models nonlinear effects, $\gamma$ is the nonlinear coefficient defined as:
\begin{equation}
\gamma=\cfrac{\omega_0 n_2(\omega_0)}{c S(\omega_0)},
\label{eq:noncoef}
\end{equation}
where $n_2(\omega_0)$ is the nonlinear refractive index ($n_2~=~2.1 \cdot  10^{-20}~\mathrm{m^2/W}$ for ZBLAN), $S(\omega_0)$ is the effective mode area of the fiber, and $c$ is the light
speed constant.
The nonlinear response function $R(t)$ is used to model Raman contributions and can be written as:
\begin{equation}
\begin{gathered}
R(t)=(1-f_{\mathrm{R}})\delta(t)+f_{\mathrm{R}}h_{\mathrm{R}}(t)\\
= (1-f_{\mathrm{R}})\delta(t)+f_{\mathrm{R}}\cfrac{\tau^2_1+\tau^2_2}{\tau_1\tau^2_2}\exp{(-t/\tau_2)}\sin{(t/\tau_1)}\Theta(t),
\end{gathered}
\label{eq:ramancont}
\end{equation}
where $h_{\mathrm{R}}(t)$ is the Raman response function, $f_{\mathrm{R}}$ is its fractional contribution (e.g. $f_{\mathrm{R}}~=~0.062$ for ZBLAN fiber), $\Theta(t)$ is the Heaviside step function, $\delta(t)$ is the Dirac delta dunction, $\tau_1$ is the Raman period, $\tau_2$ is defined by the damping time of the network~\textendash~both parameters are selected to provide a good fit to the actual Raman-gain spectrum.

\subsection{Sub-section 1}

\lipsum[1-3]


\end{document}
